% !TEX TS-program = pdflatex
\documentclass[11pt]{article}
\usepackage[margin=1in]{geometry}
\usepackage{palatino}
\usepackage{tcolorbox}
\tcbuselibrary{skins, breakable}
\usepackage{amsmath, amssymb, microtype}
\usepackage[svgnames]{xcolor}
\usepackage{enumitem}
\usepackage{booktabs}
\usepackage{tabularx}
\usepackage{parskip}
\usepackage{titlesec}
\usepackage{float}
\usepackage{nicematrix}
\usepackage[ruled,vlined]{algorithm2e}

% Custom Colors
\definecolor{thoughtblue}{RGB}{225,238,250}
\definecolor{questionyellow}{RGB}{255,250,205}
\definecolor{strategypurple}{RGB}{245,230,255}

% Section Formatting
\titleformat{\section}{\Large\bfseries\sffamily}{}{0em}{}
\titleformat{\subsection}{\large\bfseries\sffamily}{}{0em}{}
\titlespacing*{\section}{0pt}{1.5em}{0.8em}
\titlespacing*{\subsection}{0pt}{1.2em}{0.6em}

% Thought Process Boxes
\newtcolorbox{thoughtbox}{
  colback=thoughtblue,
  colframe=RoyalBlue,
  arc=3pt,
  boxrule=1pt,
  left=6pt,
  right=6pt,
  top=6pt,
  bottom=6pt,
  fontupper=\itshape
}

% Strategy Framework Box
\newtcolorbox{strategybox}{
  colback=strategypurple,
  colframe=DarkOrchid,
  arc=3pt,
  boxrule=1pt,
  left=6pt,
  right=6pt,
  top=6pt,
  bottom=6pt,
  fontupper=\normalfont
}

% Question/Note Formatting
\newcommand{\researchquestion}[1]{%
  \vspace{0.5em}\noindent%
  \colorbox{questionyellow}{\parbox{\dimexpr\textwidth-2\fboxsep}{%
  \textbf{Research Question:} #1}}\vspace{0.5em}%
}

% Custom Table Formatting
\newcolumntype{Y}{>{\raggedright\arraybackslash}X}
\renewcommand{\arraystretch}{1.2}

\begin{document}

\begin{center}
\sffamily
{\Huge Cabbage Merchant Strategy Analysis}\\[0.5em]
{\large SPARC Application Thought Journal}\\[1em]
Boden Moraski | \today
\end{center}

% Introduction
\section*{Introduction}
\begin{thoughtbox}
"Starting a business as a cabbage merchant involves balancing customer satisfaction, inventory management, and cost control. Let's work through the dynamics to find an optimal strategy."
\end{thoughtbox}

% Problem Statement
\section*{Problem Statement}
Every morning, a customer orders one cabbage. If you accept the order, you promise to deliver within three days for \$4. Failing to deliver results in a refund and an additional \$1 apology. You can also decline orders.

Each night, you can order any number of cabbages from a farmer at \$1 per cabbage. Cabbages arrive after two days. However, farmers might cancel entire orders one day before delivery without a refund. Any remaining cabbages spoil at the end of each day.

Our goal is to determine the best strategy to maximize profit while managing risks associated with order cancellations and spoilage.

% Initial Exploration
\section*{Initial Exploration}
\begin{thoughtbox}
"Let's simplify the situation by ignoring overlapping orders and daily fluctuations initially. We'll focus on the basic profit equation."
\end{thoughtbox}

\subsection*{Core Assumptions}
\begin{itemize}[leftmargin=*,label=\color{DarkSlateBlue}\textbullet]
  \item \textbf{Constant Cancellation Rate}: Probability \( P \) that a farmer cancels an order.
  \item \textbf{Independent Failures}: Each order cancellation is independent of others.
  \item \textbf{No Inventory Carryover}: Unsold cabbages spoil daily and cannot be carried over.
\end{itemize}

\subsection*{Profit Equation Breakdown}
\begin{equation*}
\mathbb{E}[\text{Profit}] = \underbrace{4(1-P^k)}_{\text{Revenue}} - 
\underbrace{k}_{\text{Cost}} - 
\underbrace{5P^k}_{\text{Penalty}}
\end{equation*}
Where:
\begin{itemize}
  \item \( k \) = Number of cabbages ordered
  \item \( P^k \) = Probability that all \( k \) cabbages are canceled
\end{itemize}

% Algorithm for Order Decision
\section*{Order Decision Algorithm}
\begin{thoughtbox}
"To systematically decide whether to accept or decline an order, we'll develop an algorithm that considers current inventory, cancellation probabilities, and cost factors."
\end{thoughtbox}

\begin{algorithm}[H]
\caption{Order Acceptance Decision Algorithm}
\KwIn{Current Inventory $I$, Cancellation Probability $P$, Current Cost $c$}
\KwOut{Decision to Accept (Yes/No)}
\If{$I \geq 1$}{
    \Return Accept\;
}
\Else{
    Calculate expected profit for accepting: $\mathbb{E}[\text{Profit}_{\text{accept}}] = 4(1-P) - 1 - 5P$\;
    Calculate expected profit for declining: $\mathbb{E}[\text{Profit}_{\text{decline}}] = 0$\;
    \If{$\mathbb{E}[\text{Profit}_{\text{accept}}] > \mathbb{E}[\text{Profit}_{\text{decline}}]$}{
        \Return Accept\;
    }
    \Else{
        \Return Decline\;
    }
}
\end{algorithm}

\subsection*{Weather-Dependent Strategy}
When weather conditions affect cancellation probabilities, we need a more sophisticated approach:

\begin{algorithm}[H]
\caption{Weather-Dependent Strategy Algorithm}
\KwIn{Current Inventory $I$, Cancellation Probability $P$, Weather Condition $W$}
\KwOut{Decision to Accept (Yes/No)}
\If{$W = \text{Bad}$}{
    $P' = f(W)$ \tcp*{Adjust cancellation probability based on weather}
}
\Else{
    $P' = P$\;
}
\If{$I \geq 1$}{
    \Return Accept\;
}
\Else{
    Calculate expected profit for accepting: $\mathbb{E}[\text{Profit}_{\text{accept}}] = 4(1-P') - 1 - 5P'$\;
    Calculate expected profit for declining: $\mathbb{E}[\text{Profit}_{\text{decline}}] = 0$\;
    \If{$\mathbb{E}[\text{Profit}_{\text{accept}}] > \mathbb{E}[\text{Profit}_{\text{decline}}]$}{
        \Return Accept\;
    }
    \Else{
        \Return Decline\;
    }
}
\end{algorithm}

% Strategic Decision Mathematics
\section*{Strategic Decision Mathematics}
\begin{thoughtbox}
"Under what conditions should I order one cabbage versus two? Let's compare the expected profits."
\end{thoughtbox}

\subsection*{Expected Profit Comparison}
For a given cancellation probability \( P \):
\begin{itemize}[label=\color{DarkSlateBlue}\textbullet]
  \item \( k=1 \): \(\mathbb{E}_1 = 4(1-P) - 1 - 5P = 3 - 9P\)
  \item \( k=2 \): \(\mathbb{E}_2 = 4(1-P^2) - 2 - 5P^2 = 2 - 9P^2\)
\end{itemize}

\begin{strategybox}
\subsubsection*{Determining the Optimal \( k \)}
To find when ordering two cabbages is better than one:
\begin{align*}
2 - 9P^2 &> 3 - 9P \\
9P^2 - 9P + 1 &< 0 \\
P &= \frac{9 \pm \sqrt{81 - 36}}{18} = \frac{9 \pm \sqrt{45}}{18} \\
P &\approx \frac{9 \pm 6.708}{18} \Rightarrow P \approx 0.127,\ 0.873
\end{align*}
\end{strategybox}

\subsection*{Profitability Constraints}
Based on the above calculations:
\begin{itemize}[label=\color{DarkSlateBlue}\textbullet]
  \item **Single Order (\( k=1 \))** is profitable if \( P < 0.127 \).
  \item **Double Order (\( k=2 \))** is profitable if \( 0.127 \leq P < 0.873 \).
  \item Ordering more than two cabbages may not be profitable within this model.
\end{itemize}

\subsection*{Numerical Case Studies}
\begin{table}[H]
\centering
\begin{tabularx}{0.95\textwidth}{lYYY}
\toprule
\textbf{Scenario} & \( P=0.1 \) & \( P=0.3 \) & \( P=0.5 \) \\
\midrule
\( \mathbb{E}_1 \) & \( 3 - 9(0.1) = 2.1 \) & \( 3 - 9(0.3) = 0.3 \) & \( 3 - 9(0.5) = -1.5 \) \\
\( \mathbb{E}_2 \) & \( 2 - 9(0.1)^2 = 1.91 \) & \( 2 - 9(0.3)^2 = 1.19 \) & \( 2 - 9(0.5)^2 = -0.25 \) \\
\textbf{Optimal} & k=1 & k=2 & Decline \\
\bottomrule
\end{tabularx}
\caption{Expected Profit Comparison Across Different Cancellation Probabilities}
\end{table}

\begin{strategybox}
\subsubsection*{Strategy Decision Rule}
Based on the probability \( P \), the optimal number of cabbages to order \( k^* \) is:
\[
k^* = \begin{cases}
1 & \text{if } P < 0.127 \text{ (Low Risk)} \\
2 & \text{if } 0.127 \leq P < 0.471 \text{ (Moderate Risk)} \\
0 & \text{if } P \geq 0.471 \text{ (High Risk)}
\end{cases}
\]
\end{strategybox}

\subsection*{Sensitivity Analysis}
Examining how changes in \( P \) affect the expected profit:
\begin{align*}
\text{Marginal Benefit} &= \frac{\partial \mathbb{E}}{\partial k} = 4P^k\ln \left(\frac{1}{P}\right) - 1 - 5P^k\ln P \\
\text{Break-even Point} & \text{ occurs when } 4P^k\ln \left(\frac{1}{P}\right) = 1 + 5P^k\ln P
\end{align*}

% Phase Transition Matrix
\subsection*{Phase Transition Matrix}
\begin{NiceTabular}{|c|c|c|c|}[hvlines]
\hline
\RowStyle{\bfseries}Probability Range & Strategy & Revenue Aspect & Risk Aspect \\
\hline
$[0, 0.127)$ & Order 1 & $4(1-P)$ & $-9P$ \\
$[0.127, 0.471)$ & Order 2 & $4(1-P^2)$ & $-9P^2$ \\
$[0.471, 1]$ & Don't Order & 0 & 0 \\
\hline
\end{NiceTabular}

% Strategy Summary
\section*{Strategy Summary}
\begin{table}[H]
\centering
\begin{tabularx}{0.95\textwidth}{lYY}
\toprule
\textbf{Condition} & \textbf{Optimal Strategy} & \textbf{Key Consideration} \\
\midrule
Low Risk (\( P < 0.127 \)) & Single Order (\( k=1 \)) & Minimal cost with low risk of cancellation \\
Moderate Risk (\( 0.127 \leq P < 0.471 \)) & Double Order (\( k=2 \)) & Balancing additional cost against increased reliability \\
High Risk (\( P \geq 0.471 \)) & Decline Orders & Avoiding losses from high cancellation rates \\
Correlated Failures & Conditional Acceptance & Incorporate weather monitoring and adjust orders accordingly \\
\bottomrule
\end{tabularx}
\caption{Strategic Decision Matrix for Cabbage Merchant Scenario}
\end{table}

% Future Research Directions
\subsection*{Future Research Directions}
\begin{itemize}[leftmargin=*,label=\color{DarkSlateBlue}\textbullet]
  \item Explore reinforcement learning techniques for adapting strategies in dynamic environments
  \item Validate models with real-world data from urban farming operations
  \item Extend economic models to include risk-transfer instruments like insurance to mitigate potential losses
\end{itemize}

% Conclusion
\section*{Conclusion}
\begin{thoughtbox}
"After analyzing various strategies and considering factors like cancellation probabilities and weather conditions, we've developed a comprehensive approach to maximize profitability while minimizing risks."
\end{thoughtbox}

Through the strategic decision rules and adaptive frameworks outlined, a cabbage merchant can effectively balance inventory management with customer satisfaction. By dynamically adjusting orders based on real-time data and weather conditions, it's possible to enhance operational efficiency and respond adeptly to market fluctuations. Future research and real-world validation will further refine these strategies, ensuring sustained success in the competitive market of urban farming.

\end{document}